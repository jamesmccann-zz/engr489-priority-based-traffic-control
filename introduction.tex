\chapter{Introduction}

Transportation is fundamental to the modern global economy, encouraging growth and job creation, trade, and quality of life. The success of road transportation and steady increase in the number of individual vehicles in use has resulted in an ongoing strain on societies to provide infrastructure capable of satisfying demand. 

Widespread utilisation of road transport for commercial and private use has lead to significant traffic congestion in densely populated, urban environments as demand outstrips the capacity of roading networks \cite{euro2011whitepaper,papa2003review}. As the complexity of road networks increases, so too does the need for control systems to safely and efficiently manage access to shared road by competing flows of traffic. 

Intersections with signal controls allow competing traffic flows to independently make use of the limited capacity of intersecting sections of  two or more roads. To avoid collisions, allowing traffic to flow through a controlled intersection requires stopping all competing traffic flows also demanding the intersection. As a result, controlled intersection delay is one of the most significant causes of congestion costs in urban road networks.

\section {Motivations}

Urban congestion is a significant problem facing the New Zealand transportation industry. In 2013, an independent consultation commissioned by the New Zealand Transport Agency (NZTA) found that the increase in transport cost due to congestion within Auckland City could be as high as 1.2 billion dollars annually when compared to freely flowing traffic \cite{wallis2013costs}. Modeling the cost of individual vehicle journeys through controlled intersections in a road network could allow for traffic systems to make efforts to minimise these congestion costs.

% where does this sentence go?
The New Zealand Ministry of Transport (MoT) 2008 transport strategy identifies affordability and efficiency as two primary goals of transportation development over the next three decades and recommends future congestion management strategies should make more efficient use of existing network capacity without the need to add expensive new infrastructure. Improving the effectiveness of traffic signal controls at road intersections has potential benefits for all controlled intersections in New Zealand, at significantly lower costs than infrastructure changes \cite{mot2008strategy}.

\section{Problem}

Modern intersection signal controllers seek to minimise delay by responding to vehicle demand at each incoming link. Adaptive traffic control systems, such as the Sydney Coordinated Adaptive Traffic Control System (SCATS), operated at all controlled intersections on New Zealand cities and highways; adaptively increment phase plans in response to near-real time traffic conditions and are successful for reducing delay within high demand road networks \cite{lowrie1982scats,akcelik1998evaluation,wolshon1999scats}.

Existing traffic control systems, such as SCATS, are limited to minimising the number of queued cars or average delay at an intersection. When an approaching vehicle is stopped at a controlled intersection, a cost is absorbed by the occupants or owners of the vehicle. Costs incurred may be caused by the physical characteristics of a vehicle, for example: a stopped vehicle must use more fuel to accelerate back to a cruise speed; or by the impact of the delay on the vehicle occupants in terms of added commuting time. To reduce the costs of congestion on road networks, traffic control systems should be designed to consider the potential costs of each vehicle within the network.

As an example of this problem, consider a common "cross-roads" intersection, with two competing approaches. If a large, commercial freight vehicle running late for a ferry and a small family car returning home from a shopping trip are approaching the intersection on two competing roads, who should be given right of way? There is significantly more cost incurred if the truck is forced to stop at the intersection, including cost of fuel required to accelerate and the potential of being late for the ferry and missing a shipment. In a traditional vehicle actuated or adaptive traffic control system, there is no guarantee on who will be given the opportunity to pass first. The traffic controller is not influenced by the potential costs of stopping approaching traffic and, depending on the current signal phase timing, it is likely that both vehicles are forced to stop, or the truck is forced to stop. 

This project presents a new method for adaptive traffic control that considers individual vehicles approaching or waiting at an intersection based on a dynamically calculated \emph{priority} value, calculated using properties of each vehicle that can be communicated to a traffic controller using wireless devices embedded in vehicles. Vehicular Ad-hoc Networks (VANETs) that allow vehicles to communicate with other vehicles and infrastructure using short-range wireless technologies are an area of increasing research interest and offer new opportunities for adaptive traffic signal control \cite{adaptive2007grad,nadeem2004trafficview,yang2004vehicle}.

\section{Objectives}

This project involves three core objectives. The chapters following will discuss each of the project developments and contributions with respect to the objectives identified here.

Firstly, design of a priority model to estimate the costs incurred by the journey of an individual vehicle through a road network. This objective is required to establish a basis for estimating and minimising cost values for a priority-aware traffic control system. A priority model is expected to assign a relative measure of the priority of every vehicle in a road network, based on the cost of stopping or delaying the vehicle. Vehicle priority modelling allows for consideration of a wide range of vehicle and motorist properties, including size and weight, fuel efficiency, number of passengers, individual passenger urgency, and purpose of transit. 

Secondly, design and implementation of a new method of traffic control using the developed priority model to determine phase allocations. The traffic control system designed and implemented by this project is required to be aware of the location, trajectory, and estimated journey costs for vehicles approaching a control intersection. Inter-vehicular, short range, ad hoc communication is proposed between vehicles and a traffic controller in order to receive responsive, real-time information about the location and properties of vehicles approaching an intersection. 

Finally, to evaluate the performance of the developed traffic control method, a software simulation tool is required. The tool developed during this project should be capable of measuring costs incurred by vehicles traveling through a simulated network, and report the performance of a traffic control system using aggregated cost calculations over an extended time period. The developed traffic control method should also be compared to existing traffic control strategies to determine the relative performance with respect to cost reduction.


