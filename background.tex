\chapter{Background and Related Work}

This chapter presents an introduction to adaptive signal control techniques used in existing, real-world traffic control systems, and a brief review of literature related to lookahead based control techniques. 

\section{Adaptive Signal Control Optimisation Techniques}

This section provides a review of published literature on existing implementations of adaptive traffic control systems in use globally and within New Zealand, and identifies the benefits and limitations imposed by the use of these systems.

\subsection{Sydney Coordinated Adaptive Traffic System}

SCATS is a centralised, coordinated, adaptive traffic control system \cite{lowrie1982scats}. In New Zealand, all controlled intersections operate on isolated control or within a coordinated network under the Sydney Coordinated Adaptive Traffic System (SCATS). SCATS operations within New Zealand are controlled by the New Zealand Transport Association (NZTA), for state highways and inter-city motorways; and local body councils where appropriate. sSCATS operates on a networked computer with two-way communication to individual SCATS connected traffic controllers over broadband (or modem) connections. SCATS interfaces with roadside traffic signal control units, requesting phase times, skipping phases, or adjusting cycle lengths on an adaptive basis. A traffic control engineer can monitor traffic demand and flow rates for an intersection and manually adjust SCATS calculated phases or cycle lengths if required.

SCATS incrementally adjusts the planned phase times of a traffic signal controller by responding to traffic data collected by the signal controller during the previous cycle. Inputs to SCATS from each individual controller include the number of vehicles and flow rate per each intersection approach, the expected and actual phase times, and the degree of saturation for the intersection.  The SCATS system calculates and requests phase times and cycle lengths to minimise the degree of saturation of an intersection, defined as the ratio of effectively used green time to total available green time \cite{wolshon1999scats}. The proportion of effectively used green time is typically increased using longer cycles and higher split times for high demand approaches. 

In a coordinated traffic control system, emphasis is given to ensuring that green times between two nearby intersections are scheduled in such a way as to allow for synchronised green phases, preventing vehicles arriving from an upstream intersection being required to stop downstream. The effect of this synchronisation is colloquially known as a "corridor of green" and will be familiar to most New Zealand inner city motorists. SCATS intersections are organised into groups called subsystems, typically based on proximity. A traffic control engineer identifies a critical intersection within each subsystem in a road network. The cycle time is optimised for the critical intersection and neighbouring intersections adopt the same cycle time to provide naive coordination of phases and ensure undersaturation of the critical intersection \cite{kilby2010rta}.

%The SCATS system works well for intersection sites with well established traffic flow periods, for example, highway intersections that have relatively even demand with single morning and evening peaks. 
In practice, SCATS is limited by the ability to adjust timings only at the conclusion of a cycle, and the relatively small incremental adjustments made between cycles. During peaks of high intersection demand, the time for a cycle length increases, typically as long as 120 seconds or higher. Adjustments made to phase and cycle times at the end of each cycle are typically within the range of 5\%-10\%. As a result, SCATS can be slow to respond to disruptive periods of high demand and requires manual intervention from traffic engineers to handle such situations, for example, sporting or musical events with large numbers of fans entering and leaving a stadium at the same time. 

\subsection{Splits-Cycle-Offsets-Optimization-Technique}

The Splits-Cycle-Offsets-Optimization-Technique (SCOOT) is an adaptive traffic control system first developed and deployed in the United Kingdom.

The primary objective of SCOOT is to minimise the sum of the average traffic queues in an area. SCOOT modelling is based on construction of so called "cyclic flow profiles", online relative to real-time demand measured by detectors upstream of an intersection. A cyclic flow profile is a measure of a one-way flow of vehicles past a point (e.g. stop line) during a time step of a signal cycle. The use of cyclic flow profiles generated online in respond to actual traffic demand is promoted as an advantage of SCOOT over fixed-plan adaptive systems such as SCATS, as SCOOT does not require a traffic engineer to predetermine a set of plans to model traffic flow or congestion at an intersection \cite{bell1992future,robertson1991optimizing}.

The SCATS and SCOOT control systems are limited by the reliability of communication links between signal controllers and the central optimiser. If communication is interrupted or lost, signal controllers will revert to a fallback mode, using predefined plans designed by a traffic engineer. In order to maintain integrity of a network in the event of communication loss, fallback plans are updated regularly by traffic control engineers using historical time of day data collected over a reasonable time period, a costly operation which requires continuous maintenance. In addition, SCATS and SCOOT both rely on the use of inductive loops installed within the pavement of a road at intersection stop-lines or at an upstream location, which must be replaced each time the surface of the road is maintained  \cite{bell1992future}.

\section{Lookahead Based Control}

Recent work has explored alternatives to phase based control. \citeasnoun{van2008movement} present a "movement-based" lookahead optimisation algorithm that allows vehicle demand to pass through an intersection in distinct \emph{movements}, which represent a passage of traffic from an approach lane to another exit lane, rather than structured phases or stages which are typically predefined sets of one or more movements. Movement-based control allows for clearance of more approaches by starting and ending individual movements of non-conflicting movements of traffic rather than a entire phases. The use of movements in control optimisation reduces the search space required by a decision tree, allowing a signal controller to look ahead to a N-second event horizon.

% need more on ALLONS-D here, maybe needs it's own section
Similar algorithms implementing lookahead optimisation techniques have also been explored using traditional phase control. ALLONS-D is a adaptive lookahead control algorithm that uses a lookahead component to minimise total delay at an intersection, with significantly improved results over Webster's method for fixed-time phase implementations \cite{porche1996allonsd}. Unfortunately, a lack of  lack of comparison with traffic actuated or adaptive controllers limit the results of the ALLONS-D evaluation and more work is needed in this area to compare with realistic traffic controllers.

Before computer controlled traffic systems were commonplace on established road networks, \citeasnoun{miller1963computer} described a potential technique for delay minimisation using a computer to estimate the cost of delay over a lookahead window. The technique is based upon inspecting the movement of vehicles at a traffic network every $h$ seconds (e.g. 2 seconds), and determining whether intersection signals should be changed immediately or left as they are to continue the current phase. If the current phase is left to continue, the change will be reassessed in another $h$ seconds at the next iteration of the algorithm. This algorithm is designed to minimise the overall delay time for the network and includes consideration of the additional delay times incurred if a phase change is delayed or changed for each approaching road link, for example, if a phase is not changed then additional delay is incurred by vehicles waiting at a red signal. The change is scheduled whenever the cost of potential delay for vehicles approaching a green signal is less than the cost of incurred delay for vehicles waiting at red signals. 

% do something with this shit 
%Signal control optimisation has been well researched with respect to minimising the total delay for all vehicles at a controlled intersection. Webster's method \cite{webster1958}, is a widely adopted and researched method for estimating optimal cycle length for minimal delay at a controlled intersection. For a saturated intersection, Webster's method provides an optimal cycle length with respect to minimal time lost between phases, between flows of opposing traffic. 







