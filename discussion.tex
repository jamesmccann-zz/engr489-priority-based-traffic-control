\chapter{Summary}

\section{Applicability}

A considerable effort has been made to ensure the validity and applicability of the results generated by this project. Design decisions related to the traffic priority model used to estimate journey costs; software design of PBTSim for simulation; and design and implementation of the PBTC control strategy, Vehicle Actuated control strategy, and SCATS representation; have been carefully considered to model realistic traffic scenarios. Threats to validity are based upon the assumptions made when designing estimate models, and representation of the traffic composition and SCATS behaviour during experimental evaluation.

\subsection{Priority Modeling for Cost Estimation}

The priority model designed during this project was required for the design and development of the PBTC system, and evaluation of the performance of the three control strategies tested during experimental evaluation. Cost of delay and cost of stopping are the primary components of this model. These costs are quantifiable and based on existing implicit costs incurred by motorists on real-world road networks. 

To calculate the cost of delay, a model has been designed based on the New Zealand Transport Agency recommended figure of \$26.20 New Zealand Dollars per vehicle hour, as opposed to an arbitrarily designed numerical value with no research as the basis of quantification. For the purposes of this project, this value is assumed to be correct for all vehicles on New Zealand roads. 

To calculate the cost of stopping a vehicle, a physics based fuel consumption model has been designed based on the kinetic energy required to accelerate a vehicle to a given speed, and the amount of speed effectively ``lost'' when a vehicle is forced to stop their journey at a controlled intersection. This model is limited by lack of consideration of different engine efficiencies, power to consumption ratios of different vehicle gears, and driver behaviours; all of which can improve or deteriorate the costs of fuel consumption involved in a forced stop. The separation of simulation vehicles into fixed property light or heavy classes is a significant simplification which doesn't accurately model realistic traffic composition on New Zealand roads. It is expected that more sophisticated modeling may influence the cost values produced by experimentation, but consistently, so the performance of each of the control strategies will not be affected.

\subsection{SCATS Log Data}

 This project relies on the data contained in log files produced by the SCATS system operating in Wellington City and provided by the Wellington City Council for simulation of traffic volume and experimental evaluation of the performance of the SCATS control strategy with regards to delay and stopping cost incurred with prioritised traffic.
 
The use of logged SCATS data introduces threats to the validity of experimental results. The SCATS log files are poorly labeled and can contain up to four detectors per approach. It is often unclear exactly which detectors are recording logged data without specific knowledge of the configuration on a per intersection basis. The software package designed to parse SCATS logs within PBTSim assumed that all approach data for each labelled approach identified of interest was relevant, with the exception of the Courtenay-Tory intersection where this behaviour was overridden to exclude data that was known to relate to the omitted turning lanes of both streets. The accuracy of detector data in the SCATS log file for the Karo-Victoria intersection should be doubted, as the SCATS representation consistently performed worse on this intersection compared to Vivian-Victoria and Courtenay-Tory across all evaluation metrics, and the measured flow rate at Victoria Street was significantly less than expected.

The SCATS log data is also used to set the phase times for the SCATS system during simulation. While these phase durations can be expected to be correct, because vehicle arrivals are simulated using the Poisson process there is no certainty that the vehicles present at any instantaneous moment of simulation are reliably reflective of the real-world demand that the SCATS system responded to. More robust evaluation is suggested as an area of future work for this project.
 
\section{Future Work}

Future work related to this project is needed to develop more sophisticated vehicle priority models, representative vehicle-infrastructure communication strategies, extended simulation capabilities, and enable more explicit comparison between the PBTC control system and real-world traffic control systems.

In this project, we have presented a simplified, physics-based consumption model for estimation of stopping costs incurred by individual vehicles. Future work in this area would improve the relevance of a predictive fuel consumption model and stopping cost calculation for a wider range of vehicle classes and engine types (e.g., electric hybrids, small trucks, etc.). Vehicle and traffic controller communication has also been simulated with the assumption that a network infrastructure capable of handing this capability is economically and technically feasible. We hope the results of this project encourage further development in this area.

Due to limitations of the Movsim simulator, this project has restricted evaluation of the PBTC control system to two-phase intersections with straight-through traffic only. Future work in this area is needed to extend the simulation software to handle turning decisions and bi-directional traffic. We expect the PBTC control algorithm can be successfully applied to more complex intersection geometries and phase configurations with positive results.

Finally, direct comparison between the PBTC control algorithm and industry alternatives currently in use for real-world road networks is desirable. This would require a collaborative research and design effort by commercial entities providing such systems.

\section{Outcomes}

Delayed travel and forced vehicle stops are inevitable when opposing flows of traffic share demand over an intersection of two links within a wider road network, adding unseen implicit costs to travel for road users. The PBTC system is designed to reduce the overall costs of delay, and costs of stopping for vehicles traveling through a controlled intersection; using a vehicle priority model to evaluate economic costs incurred to individual vehicles within a simulation environment.

The results of experimental evaluation, presented in Chapter ~\ref{chapter:evaluation}, show the PBTC control algorithm is able to achieve significantly reduced incurred delay and stopping costs, when compared to the Vehicle Actuated strategy and SCATS representation developed during this project, for all vehicles traveling through the simulated intersections used for experimental evaluation.

The PBTC system achieves reduced delay times, without significant increases to the cost of stopping for all vehicles approaching an intersection, while prioritising traffic based on urgency and reducing the total trip time for vehicles whose transit is deemed to be urgent. By considering the urgency, cost of delay, and cost of stopping vehicles approaching an intersection, the PBTC control algorithm can make cost-minimising decisions regarding allocation of green time to opposing flows of traffic. Once the decision to make a phase change has been made, the lookahead capability of the PBTC control algorithm is designed to find a cost-minimal extension of the currently displayed green time that will minimise this change with respect to the cost of delaying queued vehicles, and cost of stopping and delaying approaching vehicles.
